\documentclass[11pt, oneside]{jarticle}   	% use "amsart" instead of "article" for AMSLaTeX format
\usepackage{geometry}                		% See geometry.pdf to learn the layout options. There are lots.
\geometry{letterpaper}                   		% ... or a4paper or a5paper or ... 
%\geometry{landscape}                		% Activate for rotated page geometry
%\usepackage[parfill]{parskip}    		% Activate to begin paragraphs with an empty line rather than an indent
\usepackage{graphicx}				% Use pdf, png, jpg, or eps§ with pdflatex; use eps in DVI mode
								% TeX will automatically convert eps --> pdf in pdflatex		
\usepackage{amssymb}
\usepackage{amsmath}

%SetFonts

%SetFonts


\title{説明}
\date{}							% Activate to display a given date or no date

\begin{document}
\maketitle
\section{分裂方法と淘汰の方法}
細菌が死滅する確率を$p(0 \le p \le 1) $とする。\\
一つ一つの細菌は1秒毎に$p$の確率で死滅、$1-p$の確率で分裂のどちらか一方の行動を取るものとする。\\
(./Assets/Scripts/Bacteria.cs 21行目 〜 34行目)\\

\section{数学的根拠}
上記の方法で全体としても$p$の割合だけ死滅するかは確率変数の加法定理を用いて説明できる。\\
個体数を$n(n>=1)$死滅する確率を$p$とおく。\\
$k=1,2, ...... , n$に対して確率変数$X_k$を次のように定める。\\
%\subsection{}
\[
	X_k = 
	\begin{cases}
		1 & (k\mbox{番目の細菌が死滅した })\\
		0 & (k\mbox{番目の細菌が死滅しなかった})\\
	\end{cases}
\]
$X_k$の期待値$E(X_k)$は\\
\begin{eqnarray*}
	E(X_k) & = & 1 * p + 0 * (1-p)\\
		& = & p
\end{eqnarray*}
死滅する細菌の数を$X$とおくと、$X=X_1 + X_2 + ...... + X_n$であるから期待値の加法定理より
\begin{eqnarray*}
	E(X) & = &  E(X_1) + E(X_2) + ...... + E(X_n)\\
		& = & p + p + ...... + p\\
		& = & np
\end{eqnarray*}
よって全体でnp匹死滅する。\\
元々はn匹いたから死滅する確率は\\
\begin{eqnarray*}
	\frac{np}{n} = p 
\end{eqnarray*}
となり、全体でも$p$の割合で死滅する。
\end{document}  